\documentclass{article}
\usepackage{tikz}
\usetikzlibrary{arrows.meta, fit}
\usepackage{booktabs}
\usepackage[table]{xcolor}
%stack overflow https://tex.stackexchange.com/questions/627708/tikz-how-to-put-tables-within-arbitrarily-placed-nodes
\NewDocumentCommand\mytableat{m m +m m m}{%
    % main node, will contain the table
    \node [shape=rectangle,fill=#5, align=center](#1-t) at (#4) {
         
        #3
            
    };
    \node [align=center, anchor=south](#1-c) at (#1-t.north) {#2};
    % this is the node encompassing both --- use inner sep to create
    % a white "space" around it
    \node [fit=(#1-t) (#1-c), inner sep=3mm](#1){};
}
\title{Jain, Gea paper summary}
\author{Mackenzie Norman\\ mnor2@pdx.edu}

\begin{document}
\maketitle

\section{Algoritihim details and rough implemenation vision}
Since the paper is paywalled (why a paper from the mid 90's is paywalled is beyond me) I will attempt to quickly summarize the operators and how the paper uses a 1d array to represent a chromosone. 

\subsection{Genomic Representation}
The design space is initially discretized into a finite number of $ N x N $ cells. The paper recommends setting $N$ to the LCM of the lengths of the sides of the chips. In my experience, with a more modern machine it is feasible to discretize the space to even smaller units. (In the real world pcbs are often designed with a grid that components snap to, ideally the units of this grid would be the discretization, but it can truly be arbitrary). With the grid discretized, the space a chip takes up is represented by a number on a list. 

\begin{table}[ht]
    \centering
\begin{tabular}{|l|l|l|l|l|l|}
\hline
0 & 0 & 0 & 0 & 0 & 0 \\ \hline
0 & 2 & 0 & 1 & 0 & 0 \\ \hline
0 & 2 & 0 & 1 & 0 & 0 \\ \hline
0 & 3 & 3 & 3 & 0 & 0 \\ \hline
0 & 3 & 3 & 3 & 0 & 0 \\ \hline
0 & 3 & 3 & 3 & 0 & 0 \\ \hline
\end{tabular}

\caption{Discretized Layout}
\end{table}

This is then flattended to a 1d array (another point of improvement may be using a 2d array/vec)

[0,0,0,0,0,0,0,2,0,1,0,0,0,2,0,1,0,0,0,3,3,3,0,0,0,3,3,3,0,0,0,3,3,3,0,0]

\subsection{Genetic Operators}
Because of the problem, it is rightly noted that using traditional mutation and crossover operators would often times result in unfeasible or impossible placements. Additionally a new operator is suggested: Compaction. 

\subsubsection{Mutation}
The mutation operator has three different options. The first begins with selecting a component from a parent chromosone, removing and randomly selecting a new position in the chromosone where it can be placed. (Note can is the operative phrase here since it is possible there are no locations for it to be placed.) 

\begin{figure}[ht]
\centering
\begin{tikzpicture}[]
    \mytableat{table1}{}{%
\begin{tabular}{|l|l|l|l|l|l|}
\hline
0 & 0 & 0 & 0 & 0 & 0 \\ \hline
0 & 2 & 0 & 1 & 0 & 0 \\ \hline
0 & 2 & 0 & 1 & 0 & 0 \\ \hline
0 & 3 & 3 & 3 & 0 & 0 \\ \hline
0 & 3 & 3 & 3 & 0 & 0 \\ \hline
0 & 3 & 3 & 3 & 0 & 0 \\ \hline
\end{tabular}
        }{1, -2}{orange!0}
    \mytableat{table2}{}{%
                \begin{tabular}{|l|l|l|l|l|l|}
        \hline
0 & 0 & 0 & 0 & 0 & 0 \\ \hline
0 & 2 & 0 & 0 & 0 & 0 \\ \hline
0 & 2 & 0 & 0 & 0 & 0 \\ \hline
0 & 3 & 3 & 3 & \textbf{1} & 0 \\ \hline
0 & 3 & 3 & 3 & \textbf{1} & 0 \\ \hline
0 & 3 & 3 & 3 & 0 & 0 \\ \hline
    \end{tabular}
        }{6, -2}{blue!0}

    \draw [very thick, -Stealth] (table1) -- (table2);
\end{tikzpicture}
\caption{Move Mutation operator}
\end{figure}

The second mutation swaps 2 components. If either component cannot has overlap/out of bounds issues. these are attempted to remedied using a rotation. 
\begin{figure}[ht]
\centering
\begin{tikzpicture}[]
    \mytableat{table1}{}{%
\begin{tabular}{|l|l|l|l|l|l|}
\hline
0 & 0 & 0 & 0 & 0 & 0 \\ \hline
0 & 2 & 0 & 1 & 0 & 0 \\ \hline
0 & 2 & 0 & 1 & 0 & 0 \\ \hline
0 & 3 & 3 & 3 & 0 & 0 \\ \hline
0 & 3 & 3 & 3 & 0 & 0 \\ \hline
0 & 3 & 3 & 3 & 0 & 0 \\ \hline
\end{tabular}
        }{1, -2}{orange!0}
    \mytableat{table2}{}{%
                \begin{tabular}{|l|l|l|l|l|l|}
        \hline
0 & 0 & 0 & 0 & 0 & 0 \\ \hline
0 & \textbf{1} & 0 & \textbf{2} & 0 & 0 \\ \hline
0 & \textbf{1} & 0 & \textbf{2} & 0 & 0 \\ \hline
0 & 3 & 3 & 3 & 0 & 0 \\ \hline
0 & 3 & 3 & 3 & 0 & 0 \\ \hline
0 & 3 & 3 & 3 & 0 & 0 \\ \hline
    \end{tabular}
        }{6, -2}{blue!0}

    \draw [very thick, -Stealth] (table1) -- (table2);
\end{tikzpicture}
\caption{Swap Mutation operator}
\end{figure}

Third, a component is rotated at random. (In this algoritihim, we simplify the problem by only allowing 90, 180,270 degree rotations.)

\begin{figure}[ht]
\centering
\begin{tikzpicture}[]
    \mytableat{table1}{}{%
\begin{tabular}{|l|l|l|l|l|l|}
\hline
0 & 0 & 0 & 0 & 0 & 0 \\ \hline
0 & 2 & 0 & 1 & 0 & 0 \\ \hline
0 & 2 & 0 & 1 & 0 & 0 \\ \hline
0 & 3 & 3 & 3 & 0 & 0 \\ \hline
0 & 3 & 3 & 3 & 0 & 0 \\ \hline
0 & 3 & 3 & 3 & 0 & 0 \\ \hline
\end{tabular}
        }{1, -2}{orange!0}
    \mytableat{table2}{}{%
                \begin{tabular}{|l|l|l|l|l|l|}
        \hline
0 & 0 & 0 & 0 & 0 & 0 \\ \hline
0 & 2 & \textbf{1} & \textbf{1} & 0 & 0 \\ \hline
0 & 2 & 0 & 0 & 0 & 0 \\ \hline
0 & 3 & 3 & 3 & 0 & 0 \\ \hline
0 & 3 & 3 & 3 & 0 & 0 \\ \hline
0 & 3 & 3 & 3 & 0 & 0 \\ \hline
    \end{tabular}
        }{6, -2}{blue!0}

    \draw [very thick, -Stealth] (table1) -- (table2);
\end{tikzpicture}
\caption{Rotation Mutation operator}
\end{figure}

With all three mutations, if a chip does not ``fit'' then it is first rotated, then shifted to nearby cells, and finally moved to a random location. 

\subsubsection{Crossover}
The crossover is relatively simple. Two parents are selected $A$ and $B$. A rectangular reigon of random size is selected, and expanded to ensure no components are cut off, then in the child, the components from parent $A$ are first placed, then all remaining components that fit from parent $B$ are placed. The ones that do not fit are first checked to see if their locations in parent $A$ would be feasible and if not, a random location is selected. Parents $A$ and $B$ are then swapped for child 2.
\begin{figure}[ht]
\centering
\begin{tikzpicture}[]
    \mytableat{table1}{Parent A}{%
\begin{tabular}{|l|l|l|l|l|l|}
\hline
0 & 0 & 0 & 0 & 0 & 0 \\ \hline
0 & \cellcolor{gray!40}2 & \cellcolor{gray!40}0 & \cellcolor{gray!40}0 & 0 & 0 \\ \hline
0 & \cellcolor{gray!40}2 & \cellcolor{gray!40}4 & \cellcolor{gray!40}4 & 0 & 0 \\ \hline
0 & \cellcolor{gray!40}3 & \cellcolor{gray!40}3 & \cellcolor{gray!40}3 & 1 & 0 \\ \hline
0 & \cellcolor{gray!40}3 & \cellcolor{gray!40}3 & \cellcolor{gray!40}3 & 1 & 0 \\ \hline
0 & \cellcolor{gray!40}3 & \cellcolor{gray!40}3 & \cellcolor{gray!40}3 & 0 & 0 \\ \hline
\end{tabular}
        }{1, -2}{gray!0}
    \mytableat{table2}{\textbf{+}\\Parent B}{%
                \begin{tabular}{|l|l|l|l|l|l|}
        \hline
0 & 0 & 0 & 0 & 0 & 2 \\ \hline
0 & \cellcolor{gray!40}0 & \cellcolor{gray!40}1 & \cellcolor{gray!40}1 & 0 & 2 \\ \hline
0 & \cellcolor{gray!40}4 & \cellcolor{gray!40}4 & \cellcolor{gray!40}0 & 0 & 0 \\ \hline
0 & \cellcolor{gray!40}3 & \cellcolor{gray!40}3 & \cellcolor{gray!40}3 & 0 & 0 \\ \hline
0 & \cellcolor{gray!40}3 & \cellcolor{gray!40}3 & \cellcolor{gray!40}3 & 0 & 0 \\ \hline
0 & \cellcolor{gray!40}3 & \cellcolor{gray!40}3 & \cellcolor{gray!40}3 & 0 & 0 \\ \hline
    \end{tabular}
        }{1, -6}{blue!0}
    \mytableat{table3}{Child A}{%
        \begin{tabular}{|l|l|l|l|l|l|}
        \hline
0 & 0 & 0 & 0 & 0 & 0 \\ \hline
0 & \cellcolor{gray!40}2 & \cellcolor{gray!40}1 & \cellcolor{gray!40}1 & 0 & 0 \\ \hline
0 & \cellcolor{gray!40}2 & \cellcolor{gray!40}4 & \cellcolor{gray!40}4 & 0 & 0 \\ \hline
0 & \cellcolor{gray!40}3 & \cellcolor{gray!40}3 & \cellcolor{gray!40}3 & 0 & 0 \\ \hline
0 & \cellcolor{gray!40}3 & \cellcolor{gray!40}3 & \cellcolor{gray!40}3 & 0 & 0 \\ \hline
0 & \cellcolor{gray!40}3 & \cellcolor{gray!40}3 & \cellcolor{gray!40}3 & 0 & 0 \\ \hline
    \end{tabular}
        }{6, -2}{orange!0}
    \mytableat{table4}{Child B}{%
        \begin{tabular}{|l|l|l|l|l|l|}
        \hline
0 & 0 & 0 & 0 & 0 & 2 \\ \hline
0 & \cellcolor{gray!40}0 & \cellcolor{gray!40}1 & \cellcolor{gray!40}1 & 0 & 2 \\ \hline
0 & \cellcolor{gray!40}4 & \cellcolor{gray!40}4 & \cellcolor{gray!40}0 & 0 & 0 \\ \hline
0 & \cellcolor{gray!40}3 & \cellcolor{gray!40}3 & \cellcolor{gray!40}3 & 0 & 0 \\ \hline
0 & \cellcolor{gray!40}3 & \cellcolor{gray!40}3 & \cellcolor{gray!40}3 & 0 & 0 \\ \hline
0 & \cellcolor{gray!40}3 & \cellcolor{gray!40}3 & \cellcolor{gray!40}3 & 0 & 0 \\ \hline
    \end{tabular}
        }{6, -6}{blue!0}


    \draw [very thick, -Stealth] (table1) -- (table3);
    \draw [very thick, -Stealth] (table2) -- (table4);
\end{tikzpicture}
\caption{Crossover operator}
\end{figure}
\subsubsection{Compaction}
No details are given on the specific implementation of this operator. In my wildest dreams I would implement this with an FP approach that was encoded in the genome. A naive implementation of this is find the center of the placement, move  components towards that going componentwise from the center out. 
\subsubsection{Evaluation}
The evaluation is a normalized function of the three heuristics described in the intro of section 2. Plus any other penalty functions. this is described as $f = f_1 + f_2 + P * f_3$ with $f_1$ being placement area, $f_2$ as net length, $f_3$ being all other penalty functions. 

\subsubsection{Selection}
The paper recommends two selection methods both stemming from goldberg. An ``Expected Value(EV)'' Plan and the ``Elitist'' plan. They differ in that the EV plan uses the function to determine how many reproductions of an individual are in the next generation while in the Elitist plan the fitter individuals will presist onto the next generation.



\end{document}